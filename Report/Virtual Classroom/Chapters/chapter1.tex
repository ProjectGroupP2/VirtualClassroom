
\chapter{INTRODUCTION}
\section{Introduction of project}
          The Virtual Classroom is a collaborative teaching application to assist the students to leharn in an interactive manner. It aims to complement the efforts of teachers to integrate technology into their classrooms and link the students to the Internet in educationally productive ways and provide them a stimulating, positive and enjoyable environment to study.


  Today, most of all colleges use the standard technique for teaching i.e. teachers distributes the knowledge to the students by standing in front of them. This technique requires more time. So this drawback will overcome in this project. Our project is one of the education based tool developed on Android platform. It helps in enhancing the level as well as provides ease in spreading knowledge based information. The mode of retrieving information is Wi-Fi i.e. the connection with the established server.


	Using the android application a user can stream the video lectures present in the server with which it is connected. One needs to enter the desired IP address of the server in order to gain access to its content. The only criteria are that the given server is in the range of WiFi network. After establishing the secure connection user can see the list of the video lectures currently present in the server. When a new student or staff register himself for this application he have to register by clicking on new user link. After registration the new user is not activated. For security purpose the student or staff have to be authorized by admin. Authorized staff will able to upload the video lectures onto the server. The another facility provided to staff is record video. While giving lecture he\slash she is able to start the recording. After recording completed, the staff uploads that video on server. When video is uploaded then that particular video is saved in folder according to standard and subject.


	The additional feature of this project is bookmarking a video, by using this feature student can bookmark the video \& when next time student logged in, he\slash she can able to play the bookmarked videos. The videos which are not recorded by staff such videos can also be uploaded by staff. When student log into the application he have to choose particular subject and class. After logging, student is able to see all the video lectures present on server according to subject chosen by him\slash her. When student select a video then that video starts. The videos which are not used many times can be deleted. The authority of deleting video is only given to admin. The student account as well as staff account can be deleted by admin.	





