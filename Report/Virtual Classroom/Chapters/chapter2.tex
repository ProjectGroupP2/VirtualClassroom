\chapter{LITERATURE REVIEW}

\section{Overview of the H.264 \slash AVC Video Coding Standard in IEEE Transactions on Circuits and Systems for Video Technology}
H.264\slash AVC is newest video coding standard of the ITU-T Video Coding Experts Group and the ISO\slash IEC Moving Picture Experts Group. The main goals of the H.264\slash AVC standardization effort have been enhanced compression performance and provision of a network-friendly video representation addressing conversational (video telephony) and non conversational (storage, broadcast or streaming) applications. H.264\slash AVC has achieved a significant improvement in rate-distortion efficiency relative to existing standards. This article provides an overview of the technical features of H.264\slash AVC, describes profiles and applications for the standard, and outlines the history of the standardization process.  \cite{wiegand2003overview}

\subsection{The Architecture Design of Streaming Media Applications for Android OS}
The imperfection of the Android operating system multimedia function and the complexity of streaming media system of the development cycle are long, and efficiency is low. In order to solve those problems, the thesis designed the streaming engine layer between Linux kernel layer and application framework layer of Android platform, and constructed fast and convenient streaming media application development framework. First of all, Android operating system (OS) was analyzed, and then the media engine was added into the Android OS architecture. Based on the characteristic of the streaming media system, the steaming media engine is divided into five layers, including the user interface, data capture, and data output, codec, and network transmission layer. Through an instance of video data transmission client, the feasibility of the architecture is confirmed.\cite{zhao2012architecture}

\subsection{XML Document Parsing: Operational and Performance Characteristics}
XML Parsing starts with character conversion followed by lexical analysis that is invariant among different parsing models with syntactic analysis creates data representation based on parsing model used. Using xml parsing in �Virtual Classroom� we have included the streaming capability feature. It requires low latency and memory usage and usually parses a small portion of document sequentially without having fully fledged information of entire document structure. \cite{lam2008xml}

\subsection{Android XML Processing with the Xml Pull Parser Version 1.3}
XML stands for Extensible Markup Language and was defined 1998 by the World Wide Web Consortium (W3C).  An XML document consists of elements, each element has a start tag, content and an end tag. An XML document must have exactly one root element (i.e. one tag which encloses the remaining tags). XML differentiates between capital and non-capital letters. An XML file is called valid if it is well-formed and if it is contains a link to an XML schema and is valid according to the schema. The Java programming language provides several standard libraries for processing XML files. The SAX and the DOM XML parsers are also available on Android. The SAX and DOM parser API is on Android the same as in standard Java. SAX and DOM have their limitations, therefore it is not recommended to use them on Android. The Java standard provides also the Stax parser. This parser is not part of the Android platform. Android provides for XML parsing and writing the Xml Pull Parser class. This parser is not available in standard Java but is similar to the Stax parser.\cite{xmlpullparser}





